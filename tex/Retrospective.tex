% -*- root: ../main.tex -*-
\chapter{Retrospettiva}

\section{Processo di Sviluppo}
Durante le fasi iniziali di analisi dei requisiti e durante la Sprint 0 ci si è appoggiati al tool \textbf{Trello} per tenere traccia dei task da eseguire e degli artefatti prodotti.

In seguito si è deciso di appoggiarsi ai \textbf{Project} di \textbf{GitHub}, che mettono a disposizione un kanban virtuale come trello, ma completamente integrato con le issue di github e automatizzabile.


Quindi una volta completata la fase iniziale e definiti gli item del backlog essi sono stati inseriti in un \textbf{GitHub Project} relativo all'organizzazione GitHub creata per il progetto, consultabile all'indirizzo \url{https://github.com/orgs/Unibo-PPS-1920/projects/2}.

Sotto la stessa organizzazione sono stati creati sia il repository che contiene la relazione che quello contenente l'effettivo progetto.
Per ogni sprint è stato creato un nuovo github project in cui sono stati importati gli \textbf{item} del backlog relativi alla sprint dal project globale.

Questo approccio ha consentito di avere una visione pulita dello stato del progetto a livello alto e una visione più granulare a livello di sprint.

\section{Backlog}
Ogni item del backlog è stato formulato sotto forma di \textbf{User story}, dopo avere selezionato gli item da completare durante lo sprint planning, essi vengono tradotti in uno o più sotto-task implementativi.

\section{Iterazioni}
\subsection{Sprint 0}
A questo sprint è stato assegnato il numero 0 perchè non propriamente parte del processo di sviluppo iterativo, ma fondamentale per la produzione di solide basi per gli sprint successivi.
\subsubsection{Svolgimento}
Sono stati identificati i pattern architetturali applicabili e sviluppati in dettaglio i modelli delle parti principali del sistema
\subsubsection{Considerazioni}

\subsection{Sprint 1}
\subsubsection{Svolgimento}
In questo sprint ci si è concentrati sul setup del progetto, in particolare di \textbf{Travis CI} e di \textbf{Gradle}, l'implementazione dello scheletro modellato nello sprint precendente.
In particolare sono stati implementati i componenti principali di ECS e il game engine.
\subsubsection{Considerazioni}
\subsection{Sprint 2}
L'obiettivo di questa sprint è stato quello di mostrare una schermata all'utente, con cui possa interagire.
\subsubsection{Svolgimento}
Lato view è stata sviluppata l'architettura principale MVC, il menu principale e il canvas.
Sono stati aggiunti i 3 system principali: InputSystem, DrawSystem e MovementSystem.
\subsubsection{Considerazioni}
Questo sprint è stato terminato con un leggero ritardo dovuto alla mole di lavoro necessaria a raggiungere l'obiettivo prefissatosi di mostrare all'utente un'interfaccia con cui potesse interagire
\subsection{Sprint 3}
L'obiettivo di questa sprint è stato quello di introdurre le feature mancanti dell'esperrienza base del gioco.
\subsubsection{Svolgimento}
Sono state introdotte le componenti relative alla gestione degli scontri e l'intelligenza artificiale dei nemici, la gestione dei powerUp e la riproduzione dei suoni.
\subsubsection{Considerazioni}
Questo sprint si è rivelato più corposo del previsto, principalmente dovuto a un'astrazione non perfetta delle componenti di fisica, che hanno portato a un parziale redesign delle stesse all'introduzione del collision system.
\subsection{Sprint 4}
\subsubsection{Svolgimento}
\subsubsection{Considerazioni}
\section{Commenti Finali}