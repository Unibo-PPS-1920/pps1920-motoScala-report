\section{Ruben Ceroni}


Lo studente Ruben Ceroni, conclusa la parte iniziale di progettazione comune si è concentrato sull'implementazione dei sistemi core del gioco e della loro integrazione con il resto del sistema.

\subsection{AISystem}
L'\texttt{AISystem} si occupa di generare gli spostamenti per le entità non controllate dal giocatore dotate di intelligenza artificiale e quindi di un AIComponent, nel quale è specificata la precisione dell'intelligenza artificiale e i possibili bersagli della stessa.

Il core della logica di movimento è stato implementato in \textbf{Prolog}, attraverso il predicato \texttt{move\_avoiding(+Q,+(SX,SY),+(TX,TY),+O,-DX,-DY)} che, prendendo in input una tupla (X,Y) di partenza e una di destinazione, fornisce in output il vettore direzione da prendere per arrivare alla destinazione.
La precisione è influenzabile tramite l'argomento Q, che esprime di quanto verrà falsata la posizione da raggiungere fornita in input.
L'array di posizioni O è necessario ad evitare che più nemici con lo stesso bersaglio si scontrino eccessivamente tra di loro inseguendolo.
La direzione risultante calcolata viene inserita nella stessa coda gestita dall'InputSystem che quindi gestisce lo spostamento delle entità nemiche allo stesso modo dei player.

\subsection{EndGameSystem}
L'EndgameSystem si occupa di monitorare lo stato della partita e di rimuovere le entità che escono dal bordo fornitogli in input o la cui vita abbia raggiunto lo 0 a causa delle collisioni subite.
All'eliminazione di un'entità  viene inviato tramite il mediator un \textbf{LevelEndEvent} contenente lo status della partita, l'entità eliminata e il punteggio ottenuto eliminandola.
Nel caso si tratti di un'entità nemica esso viene semplicemente rimosso dal gioco.
Mentre se si tratta di un'entità player o il player stesso rimane da solo in gioco, la partita viene considerata conclusa e viene stoppato l'engine, mentre lato view è mostrata la schermata di vittoria o di sconfitta. 
 
\subsection{PowerUpSystem}
Sono stati implementati 3 differenti tipi di powerUp

L'attivazione di un powerUp è controllata dal \textbf{CollisionSystem}, che alla collisione di una entità player con un powerUp ne inserirà il riferimento all'interno del \textbf{PowerUpComponent} del powerUp

Il comportamento specifico del singolo powerUp è specificato invece all'interno del suo \textbf{PowerUpEffect}, tramite una funzione che specifica come modificare il valore del componente modificato.

% \subsection{DrawSystem}

\subsection{Engine e GameLoop}
L'Engine rappresenta la parte core del gioco, e si occupa di istanziare i componenti principali e di registrare le entità del livello presso il Coordinator.
Il \textbf{GameLoop} è stato implementato come un thread separato che durante la sua esecuzione chiamerà il metodo tick dell'\textbf{UpdatableEngine} fornitogli in input, che porta all'aggiornamento di tutti i sistemi e la produzione di un frame di gioco. Per mantenere costante la frequenza di aggiornamento viene calcolato il tempo trascorso per produrre il frame e nel caso sia inferiore alla durata necessaria a mantenere gli fps costanti viene fatto attendere il thread.
