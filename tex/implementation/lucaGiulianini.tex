% -*- root: ../../main.tex -*-
\section{Luca Giulianini}
Lo studente Luca Giulianini si è occupato delle seguenti macro-parti:
\begin{itemize}
	\item{\textbf{Architettura generale e mediazione MVC-ECS}}
	\item{\textbf{Architettura ECS}}
	\item{\textbf{Architettura View}}
	\item{\textbf{Continuous Integration e configurazione dei Build-Tools}}
\end{itemize}

inoltre ha contribuito al lavoro dei colleghi svolgendo una serie di \textbf{micro-task}.

\paragraph{Approfondimenti}
\label{par:approfondimenti}
Parallelamente all'esperienza di progetto sono stati approfonditi una serie di argomenti relativi al mondo di \textbf{Scala} e della \textbf{programmazione funzionale}. 
\begin{itemize}
	\item{\textbf{Monadi:}} è stato chiarito il concetto astratto di \textbf{Monade}.
	\item{\textbf{Pogrammazione funzionale:}} è stato approfondito l'insieme delle \textbf{buo-ne pratiche} relative alla programmazione funzionale. Questo è stato fatto seguento svariate \textbf{conferenze} dedicate al mondo Scala e appoggiandosi a una serie di \textbf{libri} dedicati. 
	\item{\textbf{Refined Types:}} Molto utili nella fase di \textbf{validazione statica e dinamica} di input. Attualmente non utilizzati data la natura del progetto in questione.
	\item{\textbf{Monix e Akka Streams:}} È stato approfondito il mondo della programmazione \textbf{reattiva} focalizzandosi maggiormente sulle differenze fra il framework \texttt{Monix} e \texttt{Akka Streams}.
\end{itemize}

\paragraph{Risorse}
\label{par:risorse}
Gli approfondimenti sono stati svolti grazie all'utilizzo delle seguenti risorse:
\begin{itemize}
	\item{\textbf{Libri:}}
	\begin{itemize}
		\item{\textbf{Programming in Scala:}} \cite[scalaBook:2014]{scalaBook:2014}
		\item{\textbf{Scala with Cats:}} \cite[scalaCats:2014]{scalaCats:2014}
		\item{\textbf{Scala Cookbook:}} \cite[scalaCook:2014]{scalaCook:2014}
		\item{\textbf{Functional Programming in Scala:}} \cite[functionalScala:2014]{functionalScala:2014}
	\end{itemize}
	\item{\textbf{Conferenze:}}
	\item{\textbf{Siti Web:}}
\end{itemize}


\subsection{Architettura generale e mediazione MVC-ECS}
\subsection{Architettura ECS}
\subsection{Architettura View}
\subsection{Continuous Integration e configurazione dei Build-Tools}
\subsection{Micro-task}
