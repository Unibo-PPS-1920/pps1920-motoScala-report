\section{Gyordan Caminati}

Lo studente Gyordan Caminati si è occupato prevalentemente nelle seguenti macro-aree:

\begin{enumerate}
    \item \textbf{Manager}
    \item \textbf{Multiplayer}
    \item \textbf{View}
    \item \textbf{Controller}
\end{enumerate}

\subsection{Manager}
    Lo studio e l'implementazione di tutti i \textbf{Manager} sono stati svolti individualmente.
    In particolare, lo studente si è occupato delle seguenti parti:
    \begin{itemize}
        \item File Manager
        \item Package Object
	    \item Audio Agent
	    \item Data Manager
	    \item Yaml Manager
    \end{itemize}
    
    \subsubsection{File Manager}
    Il File Manager permette di effettuare le operazioni di gestione dei file, come, ad esempio, il caricamento da Jar, la cancellazione di una cartella o la creazione di un albero delle cartelle. Il File Manager è stato realizzato come un \textit{Singleton} poiché si è ritenuto concettualmente corretto che ne esistesse solo uno, visibile da qualsiasi punto ci si trovi e che espone i metodi principali per la gestione dei file.

    \subsubsection{Package Object}
    All'interno del package File, dove risiedono diversi manager che necessitano di eseguire operazioni su file, si è ritenuto necessario introdurre un Package Object che esponesse dei metodi per facilitarne la gestione. Alcune delle operazioni da agevolare riguardavano, ad esempio, l'utilizzo pervasivo di \texttt{Path}, \texttt{File} e \texttt{Stringhe} che identificano il percorso in forma testuale. Nel \textbf{Package Object} sono inoltre presenti anche tutte le costanti
    che servono a risalire al percorso dei file testuali.
    
    L'introduzione del Package Object ha migliorato notevolmente il flusso di programmazione dato che il passaggio da \texttt{String} a \texttt{Path} è molto intuivo e naturale. Lo stesso discorso può essere applicato anche per \texttt{Path} e \texttt{File}.
    L'aggiunta di questi due metodi impliciti fornisce gratuitamente il passaggio da \texttt{String} a \texttt{File}.
        
     Data la \textbf{forte probabilità} di incappare in una \texttt{Exception} durante una qualunque operazione I/O,si è deciso di sviluppare un ultimo metodo che incapsulasse una \texttt{Try} e trasformasse l'eventuale  \texttt{Exception} in un \texttt{Boolean}, riportando successivamente su un \textbf{Log} l'accaduto.
    
    \subsubsection{Audio Agent}
    L'\texttt{Audio Agent} è il modulo preposto alla gestione dei suoni. Nonostante se ne utilizzi effettivamente uno solo si è scelto di non renderlo un \textbf{Singleton}, come è stato invece fatto con il \texttt{File Manager}. 
    Alla base di questa scelta vi è la volontà di non precludere estensioni future.
    
    Per la realizzazione dell'Agente, si è scelto di mantenere un approccio più conservativo. 
    
    L'Agent mantiene al suo interno un \texttt{Thread} che 
    manipola e riproduce effetti e musica.
    La riproduzione di musica e effetti con un alta frequenza, richiede notevoli risorse, l'unione del Thread a una coda di messaggi(poi divenuta coda di eventi grazie all'apporto fornito da Giulianini) consente un' ottima separazione dei flussi di controllo, e coesione con altri Thread.
    
    Nonostante l'implementazione attuale volga lo sguardo verso un approccio generico, l'interfaccia lascia la strada aperta a successive implementazioni, come Agent specifici per suoni ad alta qualità, mantenendo sempre un approccio molto pulito e neutrale.
    
    Il contributo di design apportato dal collega \textbf{Giulianini} ha consentito di trasformare il modello a messaggi, in un modello ad eventi. In questo modo si è riusciti a rendere il tutto più
    pulito e ad evitare un match case
    che avrebbe sporcato l'implementazione.
    
    \subsubsection{Yaml Manager}
    La serializzazione e la deserializzazione di tutti i dati \textbf{Locali} è gestita unicamente dallo \textbf{Yaml Manager}. Esso fornisce solo poche funzioni generiche, astraendo e delegando ad un livello superiore 
    la concatenazione e l'utilizzo delle stesse.
    
    La \textbf{Serializzazione} è effettuata principalmente utilizzando il formato 
    \texttt{Yaml}. Inizialmente è stato valutato anche l'utilizzo di \texttt{Json}, ma quest'ultimo risultava più prolisso e meno leggibile. Per questo motivo si è preferito scegliere Yaml.
    Questa decisione è stata guidata anche dal fatto che il \textbf{livello} (la principale fonte di serializzazione) è composto prevalentemente da una lista di \texttt{Entity} e alcuni parametri. Le liste in  \texttt{Yaml} sono particolarmente leggibili.
    
    Per la serializzazione si è scelto di utilizzare un serializzatore basato su \texttt{Jackson}, perché esso è molto veloce.
    \texttt{Jackson}, inoltre, ha consentito una scrittura pulita dei dati da serializzare.
    Normalmente per la serializzazione di una classe, sarebbe necessario l'utilizzo pervasivo di annotazioni come, ad esempio: \texttt{@JsonProperty} o \texttt{@JsonFormat}; tuttavia la definizione di dati snelli e l'utilizzo di un \texttt{PolymorphicTypeValidator} hanno consentito di serializzare e deserializzare senza la necessità di ricorrere ad annotazioni.
    
    \subsubsection{Data Manager}
    La gestione di tutti i dati è demandata al \textbf{Data Manager}
    il quale fornisce metodi diretti per il salvataggio e il caricamento di tutti i file testuali del progetto come: livelli, punteggi e impostazioni.
    Il caricamento e il salvataggio dei livelli può avvenire sia nel \textbf{File System} sia nel \textbf{Jar}.
    
    
    \subsection{View}
    Lo studente si è occupato in particolare dello sviluppo di tutte le schermate ad eccezione di \textbf{Level} e \textbf{Game}.
    Ogni schermata si appoggia sopra l'architettura base sviluppata dal collega \textbf{Giulianini}. Tutte le view utilizzano il pattern \texttt{Observer}.
    
    Di seguito, le schermate che lo studente ha realizzato e gestito:
        \begin{itemize}
	        \item Stats
	        \item Settings
	        \item ModeSelection
            \item Home
            \item Lobby
        \end{itemize}

    \subsubsection{Stats}
        Una semplice schermata che fornisce all'utente una visuale su tutti i record degli utenti registrati al sistema.
        Chiede al \textbf{Controller} la tabella dei record e la mostra.


    \subsubsection{Settings}
        La schermata adibita alle impostazioni, che appoggia sul Controller per le operazioni di IO. Essa consente al giocatore di cambiare: Nome, Difficoltà e volume di suoni e musica.
        Quando un giocatore cambia nome, può inserire anche un nome già utilizzato in precedenza; in questo caso ulteriori nuovi 
        record con quel nome sovrascriveranno il precedente.

  \subsubsection{ModeSelection}
        La schermata di ModeSelection separa la \textbf{Home} dal \textbf{Game}.
        Essa permette di scegliere se diventare un \textbf{Client} o un \textbf{Server}. Nel primo caso è necessario fornire un IP e una porta per potersi connettere al server; mentre nel secondo caso verranno scelta una porta libera e un interfaccia di rete disponibile su cui far connettere i client.
        Il controllo dell'input consente di evitare inserimenti erronei e le notifiche avvisano il giocatore nel caso in cui il suo nome sia già presente in partita oppure il server sia pieno.
        
        Nel caso in cui si cerchi di contattare un server su un indirizzo sbagliato, verrà notificato un messaggio di timeout.
        
    \subsubsection{Home}
        La schermata iniziale presenta sullo sfondo un tema a griglia e consente all'utente di navigare attraverso tutte le opzioni.
    
    \subsubsection{Lobby}
        La schermata di lobby mostra un notevole numero di informazioni; l'IP del server e la sua porta, la lista di giocatori pronti e non pronti.
        Il server ha inoltre il potere di espellere giocatori a piacimento, un eventuale giocatore espulso riceverà una notifica di allontanamento.
        
        Quando tutti i giocatori presenti nella lobby sono pronti, il server può far partire la partita. È possibile partire anche con un numero inferiore di giocatori, a patto che siano tutti pronti.
        
    \paragraph{Ogni bottone presente nelle view quando premuto accoda un evento al \texttt{SoundAgent} che lo applica.
    Le varie view sviluppate dallo studente non possono comunicare tra loro, e l'unico punto di contatto rimane sempre il controller, tramite l'interfaccia dell'observer ObservableUI.}

    \subsection{Controller}
        Il Controller contiene la struttura dati della lobby, usata per il mantenimento di una lobby durante la fase
        preliminare di una partita.
        Contiene tutta la configurazione di akka e anche un attore.
        Non vi sono mai due attori aperti in contemporanea, 
        mentre è possibile che non siano presenti attori(Single Player).
        Mantiene anche il \texttt{Mediator} che però non utilizza direttamente.
        Internamente utilizza anche il \textbf{Data Manager} che viene impiegato per il caricamento e la serializzazione di tutti i dati.
        Contiene anche l'\textbf{Engine} e il \textbf{Sound Agent}, i quali sono comandati direttamente dal controller stesso.
        
        Il controller è suddiviso in quattro Intefacce:
        \begin{itemize}
	        \item \texttt{ActorController}
	        \item \texttt{EngineController}
	        \item \texttt{ObservableUI}
        	\item \texttt{SoundController}
        \end{itemize}
        
        L'interfaccia SoundController viene estesa da EngineController, ActorController e ObservableUI; così facendo il controller fornisce un punto di ingresso al metodo \texttt{redirectSoundEvent} che consente di accodare un evento che successivamente verrà eseguito.
        
        L'interfaccia EngineController estende il SoundController; di conseguenza, oltre a fornire la possibilità di accodare \texttt{MediaEvent}, fornisce anche un getter per il \texttt{Mediator}, così l'\texttt{Engine} che lo contiene può sottoscriversi, ma può anche passare l'EngineController ad alcuni dei suoi come:
        \texttt{CollisionSystem} e \texttt{PowerUpSystem}, i quali ne faranno uso per pubblicare eventi sul \texttt{Mediator} e inviare eventi all'\texttt{AudioAgent}.
        
        L'interfaccia ActorController mette a disposizione degli attori i metodi necessari per instaurare una partita, gestirla e terminarla.
        Inoltre fornisce anche la possibilità agli attori di inviare alla lobby tramite un metodo generico una \textbf{strategia} per interrogarne e manipolarne direttamente il contenuto.
        
        Allo stesso modo ma senza ritorno alcuno, fornisce la possibilità di inviare una strategia verso la View, nella quale incapsula un evento che viene gestito poi dalla view stessa tramite \texttt{notify}.  
        
        
    \subsection{Multiplayer}
    Lo studente si è occupato prevalentemente dell'integrazione di \texttt{ClientActor} e \texttt{Server Actor}.
    
    Entrambi gli attori comunicano con l'\texttt{Actor controller} chiamando metodi su di esso; mentre tra di loro comunicano tramite scambio di messaggi. Sono presenti diversi messaggi e sono gestiti in \texttt{Behaviour} ben precisi.
    
    Dato che entrambi gli attori dovevano riportare un log nel caso in cui avessero ricevuto un messaggio fuori dal \texttt{Behaviour} corretto e che entrambi dovevano gestire il messaggio \texttt{PlayMediaMessage} (Inviato dal server per permettere al client di riprodurre suoni come collisioni e vittoria) è stato ritenuto necessario introdurre un \texttt{ActorWithLogging}. Questa classe astratta estende \texttt{Actor} e \texttt{ActorLogging} e fornisce un metodo \texttt{changeBehaviourWhitLogging} che viene utilizzato alternativamente alla \texttt{context.become}. Questa funzione concatena la \texttt{PartialFunction} \texttt{Receive} fornita in input(Behaviour) e fornisce in uscita una \texttt{PartialFunction} che è di fatto esaustiva dato che effettua un match su tutto e 
    effettua una operazione di log.
    
    Questo consente di extendere il \texttt{Behaviour} passato e ottenere per ogni caso non gestito la certezza di un log, evitando così anche errori.
    
    Inoltre è possibile introdurre un ulteriore \texttt{Behaviour} di default, che verrà chiamato dopo il classico \texttt{Behaviour}, ma prima di quello esaustivo.
    
    Nel caso del server il \texttt{Behaviour} di default introdotto ridireziona tutti i \texttt{PlayMediaMessage} verso i suoi client.
    Nel caso del Client il \texttt{PlayMedia-Message} viene inviato al \texttt{SoundAgentSoundAgent} il quale lo consumerà lanciando il media.

    \paragraph{Contributi esterni}
        Lo studente ha contribuito all'implemetazione delle strutture dati principali del model come : \texttt{MatchSetup}(lobby), \texttt{Difficulties} e \texttt{Settings}.
        Inoltre ha prodotto la prima implementazione di difficoltà, poi rifinita da \textbf{Giulianini}.
        
\paragraph{Contributi dello studente}
\begin{enumerate}
    \item \textbf{progettazione}
        \begin{itemize}
	        \item Data Manager
	        \item Audio Agent
	        \item File Manager
        	\item Lobby
        \end{itemize}
    \item \textbf{Implementazione}
        \begin{itemize}
            \item Controller
	        \item Client Actor
	        \item Server Actor
            \item Tutte le schermate di View eccetto Home e Levels
        \end{itemize}
\end{enumerate}


