\section{Gyordan Caminati}

Lo studente Gyordan Caminati si è occupato prevalentemente nelle seguenti macro-aree:

\begin{enumerate}
    \item \textbf{Managers}
    \item \textbf{Multiplayer}
    \item \textbf{View}
\end{enumerate}

\subsection{Managers}
    Lo studio e l'implementazione di tutti i \textbf{Managers} è stata svolta individualmente.
    
    In particolare, lo stesso si è occupato di:
    
    \begin{itemize}
        \item Data Manager
	    \item Audio Agent
	    \item File Manager
    \end{itemize}
    
    \subsubsection{File Manager}
    Data la necessità di amministrare vari file per le diverse informazioni (statistiche, livelli, impostazioni, musiche 
    e suoni) si è reso necessario introdurre un \textbf{File Manager}.
    Un \textit{Singleton} che fornisce i metodi base (caricamento da Jar, cancellazione cartella, creazione albero cartelle, ecc) ovunque.
        
    \subsubsection{Package object}
    In tutto il package file, dove risiedono i vari manager dedicati alla manipolazione dei file, si è fatto largamente uso di
    \texttt{Path}, \texttt{File} e \texttt{Stringhe}, dove queste ultime identificavano un percorso puramente testuale.
    A causa di questi utilizzi così pervasivo all'interno del \textbf{Package} sono stati introdotti alcuni metodi \textbf{impliciti} per agevolare la scrittura.
    Questo ha migliorato notevolmente il flusso di programmazione dato che il passaggio da \texttt{String} a \texttt{Path} è molto intuiva e naturale; lo stesso vale per \texttt{Path} e \texttt{File}.
    L'aggiunta di questi due metodi impliciti ci fornisce gratuitamente il passaggio da \texttt{String} a \texttt{File}.
    
    Nel \textbf{Package object} sono presenti anche tutte le costanti 
    di percorso dei file testuali, infine data la forte probabilità di incappare in una \texttt{Exception} durante una qualunque operazione I/O, ha portato allo sviluppo di un ultimo metodo che incapsulando una \texttt{Try} trasforma l'eventuale  \texttt{Exception} in un \texttt{Boolean}, e riporta su \textbf{Log} l'accaduto.
    
    \subsubsection{Audio Agent}
    A differenza dell'\textbf{File Manager} non è stato progettato il l'Audio Agent come un \textbf{Singleton}, nonostante si utilizzi effettivamente un solo Audio Agent.
    
    Dopo un'attenta analisi, sì è scelto un approccio più conservativo, l'Agent mantiene al suo un \texttt{Thread} che 
    manipola e riproduce effetti e musica.
    La riproduzione di musica e effetti con un alta frequenza, richiede notevoli risorse, l'unione del Thread a una coda di messaggi(poi divenuta coda di eventi grazie all'apporto fornito da Giulianini) consente un ottima separazione dei flussi di controllo, e coesione con altri Thread.
    
    Nonostante l'implementazione attuale volga lo sguardo verso un approccio generico, l'intefaccia lascia la strada aperta se pur sealed a successive implementazioni, come Agent specifici per suioni ad alta qualità, mantenendo sempre un approccio molto pulito e neutrale.
    
    \subsubsection{Data Manager}


\begin{enumerate}
    \item \textbf{progettazione}
        \begin{itemize}
	        \item Data Manager
	        \item Audio Agent
	        \item File Manager
        	\item Lobby
        \end{itemize}
    \item \textbf{Implementazione}
        \begin{itemize}
            \item Controller
	        \item Client Actor
	        \item Server Actor
            \item Tutte le schermate di View eccetto Home e Levels
        \end{itemize}
    \item \textbf{Analisi}
\end{enumerate}

\subsection{Data Manager}

\subsection{Audio Agent}
\subsection{Audio Agent}
\subsection{File Manager}
\subsection{Lobby}







e della impementazione di:

\begin{itemize}
	\item Controller
	\item Client Actor
	\item Server Actor
    \item Tutte le schermate di View eccetto Home e Levels
\end{itemize}

e della configurazione di :

\begin{itemize}
	\item Jackson per Yaml
	\item Akka??
\end{itemize}

Lo studente ha contribuito alla \textbf{progettazione} di:
\begin{itemize}
	\item Client Actor 
	\item Server Actor
\end{itemize}

inoltre ha contribuito al lavoro dei colleghi svolgendo una serie di \textbf{task implementativi}:
\begin{itemize}
	\item \hyperref[subsubsec:input_sys]{\textbf{Input System}}
	\item \hyperref[subsubsec:damage]{\textbf{Life points e collision damage}}
	\item \hyperref[subsubsec:monix_sys]{\textbf{Monix - Parallelizzazione update dei sistemi}}
\end{itemize}