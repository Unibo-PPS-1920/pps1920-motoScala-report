% -*- root: ../main.tex -*-
\chapter{Introduzione}
L'obiettivo del progetto è quello di \textbf{sviluppare} un gioco arcade ispirato a \textbf{Motos}, avendo cura di mantenere un alto grado di \textbf{qualità} non solo nel risultato dell'\textbf{implementazione} ma anche in tutto ciò che riguarda il \textbf{processo di sviluppo}.
Per questo motivo tutti i \textbf{passaggi} che porteranno al risultato finale verranno \textbf{analizzati} con attenzione e \textbf{documentati} in maniera \textbf{dettagliata}.

\section{Il gioco} L'arcade \textbf{originale} consiste in un un \textbf{campo} bidimensionale sul quale il giocatore \textbf{pilota} una navicella e ha l'obiettivo di \textbf{eliminare} le entità nemiche spingendole \textbf{fuori} dal campo, \textbf{senza} essere eliminato a sua volta.

\textbf{MotoScala} replicherà gli aspetti principali del gioco originale, arricchendoli con l'inserimento di nuove \textbf{feature}, come il concetto di \textbf{durata vitale} delle entità e il \textbf{multiplayer}.

\section{Gli obiettivi}
Durante lo svolgimento del progetto si punterà al raggiungimento di \textbf{obiettivi} riguardanti il \textbf{processo di sviluppo}:
\begin{itemize}
    \item Rispetto delle \textbf{specifiche} non opzionali ed eventualmente integrazione di quelle aggiuntive.
    \item Utilizzo appropriato di tecniche \textbf{avanzate} di \textbf{gestione} del progetto.
    \item Rispetto degli standard di \textbf{qualità} introdotti a lezione.
\end{itemize}

